\section{Learning}
We are supposed to develop a recommender system, so that it gets easier to match users with items.
In this case match users with movies they might like.
There are two paradigms of recommender systems:
\begin{itemize}
\item Content-based
\item Collaborative
\end{itemize}
\paragraph{Content-based} is to pair users with items by looking at the stuff they have liked in the past.
For movies e.g. it would genre, actors etc.

\paragraph{Collaborative} is to pair users with items by looking at other users in their network.
This is based on that users like similar items based on their peers.

\paragraph{We have chosen} the collaborative approach because the data provided only gives us the name of the movie and the release date.
The data also gives us a lot of information about which movies a user likes - so this obviously fits the collaborative paradigm.

\paragraph{If we would chose} the content-based paradigm we would need some more information about the movie - this could be obtained by using \url{imdb.com}.
Also we would need some info about what the user likes, this information could partially be obtained implicitly by looking at which movies the user likes.